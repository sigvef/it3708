\newsavebox{\zoomsinglegood}
\savebox{\zoomsinglegood}{%
\begin{tikzpicture}
\begin{axis}[
width=\textwidth/4,
ybar interval,
xtick=,% reset from ybar interval
yticklabel={\tiny $\pgfmathprintnumber\tick$},
xticklabel={\tiny $\pgfmathprintnumber\nexttick$},
]
\addplot+[hist={data={x}, bins=20, data max=250, intervals=false}]
file {measurements/1-box-7-epucks-without-unstick-random-placements-b35d4c771d6b71e88f73cf833ed8067aad97f4ff.txt};
\draw[red] (axis cs:21.344,0) -- (axis cs:21.344,18);
\end{axis}
\end{tikzpicture}
}

\newsavebox{\zoomsinglebad}
\savebox{\zoomsinglebad}{%
\begin{tikzpicture}
\begin{axis}[
width=\textwidth/4,
ybar interval,
xtick=,% reset from ybar interval
yticklabel={\tiny $\pgfmathprintnumber\tick$},
xticklabel={\tiny $\pgfmathprintnumber\nexttick$},
]
\addplot+[hist={data={x}, bins=20, data max=250, intervals=false}]
file {measurements/1-box-7-epucks-without-unstick-random-placement-bad-reposition-0c4ad0674e83c6681d4769371fe55f976d71ab72.txt};
\draw[red] (axis cs:38.336,0) -- (axis cs:38.336,18);
\end{axis}
\end{tikzpicture}
}


\begin{figure}[p]%
\centering
\subfloat[Randomly generated scenario with 1 food box and 7 epuck robots, no incentive to cooperate, without unstick behaviour. ]{{
\label{figure:first-hist}
\begin{tikzpicture}
\begin{axis}[
width=\textwidth/2,
ybar interval,
xlabel=Time to completion in seconds,
ylabel=Number of runs,
ymax=70,
xtick=,% reset from ybar interval
xticklabel={$\pgfmathprintnumber\nexttick$}
]
\addplot+[hist={data={x}, bins=20, data max=2000, intervals=false}]
file {measurements/1-box-7-epucks-without-unstick-random-placement-bad-reposition-0c4ad0674e83c6681d4769371fe55f976d71ab72.txt};
\draw[red] (axis cs:38.336,0) -- (axis cs:38.336,60) node [right=0.1cm] {38.336 s};
\draw (axis cs:1500,50) node {\usebox{\zoomsinglebad}};
\end{axis}
\end{tikzpicture}
}}%
\subfloat[Randomly generated scenario with 1 food box and 7 epuck robots, incentive to cooperate, without unstick behaviour.]{{%
\label{figure:second-hist}

\begin{tikzpicture}%
\begin{axis}[
width=\textwidth/2,
ybar interval,
xlabel=Time to completion in seconds,
ylabel=Number of runs,
ymax=70,
xtick=,% reset from ybar interval
xticklabel={$\pgfmathprintnumber\nexttick$}
]
\addplot+[hist={data={x}, bins=20, data max=2000, intervals=false}]%
file {measurements/1-box-7-epucks-without-unstick-random-placements-b35d4c771d6b71e88f73cf833ed8067aad97f4ff.txt};%
\draw[red] (axis cs:21.344,0) -- (axis cs:21.344,60) node [right=0.15cm] {21.344 s};%
\draw (axis cs:1500,50) node {\usebox{\zoomsinglegood}};
\end{axis}%
\end{tikzpicture}%

}}%





\qquad
\subfloat[Randomly generated scenario with 2 food boxes and 7 epuck robots without unstick behaviour.]{{
\label{figure:third-hist}
\begin{tikzpicture}
\begin{axis}[
width=\textwidth/2,
ybar interval,
ymax=70,
xlabel=Time to completion in seconds,
ylabel=Number of runs,
ymax=70,
xtick=,% reset from ybar interval
xticklabel={$\pgfmathprintnumber\nexttick$}
]
\addplot+[hist={data={x}, bins=20, data max=2000, intervals=false}]
file {measurements/2-boxes-7-epucks-without-unstick-random-placement-465bdfd2a08ceaf31540354378515dfb0287f40f.txt};
\draw[red] (axis cs:2000,0) -- (axis cs:2000,40) node [left] {Infinity};
\end{axis}
\end{tikzpicture}
}}%
\subfloat[Randomly generated scenario with 2 food box and 7 epuck robots with unstick behaviour.]{{
\label{figure:fourth-hist}
\begin{tikzpicture}
\begin{axis}[
width=\textwidth/2,
ybar interval,
xlabel=Time to completion in seconds,
ylabel=Number of runs,
ymax=70,
xtick=,% reset from ybar interval
xticklabel={$\pgfmathprintnumber\nexttick$}
]
\addplot+[hist={data={x}, bins=20, data max=2000, intervals=false}]
file {measurements/2-boxes-7-epucks-with-unstick-random-placement-898fc092f762f5a17d5875ae62bb9baff5335149.txt};
\draw[red] (axis cs:647.872,0) -- (axis cs:647.872,20) node [right] {647.872 s};
\end{axis}
\end{tikzpicture}
}}%

\caption{Task completion measurement time distributions for different scenarios. Each scenario was simulated 75 times. The red line indicates the median time.
Smaller, embedded graphs are zoom-ins on specific ranges of the data set of the larger set.}
\label{figure:task-completion-times}
\end{figure}
